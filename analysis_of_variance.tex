\documentclass[12pt]{article}
\title{Notes on Analysis of Molecular Variance}
\author{Xingyuan Su}
\date{\today}

\usepackage{amsmath}

\setlength{\parindent}{0pt} % don't indent the line
\begin{document}

\maketitle

\section{General Linear Model}

General linear model is expressed as:

\begin{equation*}
    \boldsymbol{Y} = \boldsymbol{X}\boldsymbol{\beta} + \boldsymbol{\epsilon}
\end{equation*}

The matrix form of the equation:

\[    
\begin{bmatrix} 
    Y_1 \\ 
    Y_2 \\ 
    Y_3 \\ 
    \vdots \\
    Y_n
\end{bmatrix}_{n \times 1}
= 
\begin{bmatrix}
    x_{11} & x_{12} & x_{13} & \cdots & x_{1p} \\
    x_{21} & x_{22} & x_{23} & \cdots & x_{2p} \\
    x_{31} & x_{32} & x_{33} & \cdots & x_{3p} \\
    \vdots & \vdots & \vdots & \ddots & \vdots \\
    x_{n1} & x_{n2} & x_{n3} & \cdots & x_{np}
\end{bmatrix}_{n \times p}
\begin{bmatrix}
    \beta_1 \\
    \beta_2 \\
    \beta_3 \\
    \vdots \\
    \beta_p 
\end{bmatrix}_{p \times 1}
+ 
\begin{bmatrix}
    \epsilon_1 \\
    \epsilon_2 \\
    \epsilon_3 \\
    \vdots \\
    \epsilon_n
\end{bmatrix}_{n \times 1}
\]

The goal of general linear model is to find $\boldsymbol{\beta}$ that minimizes 
the distance between $\boldsymbol{Y}$ and $\boldsymbol{X\beta}$. 


\section{Analysis of Molecular Variance}
The model was developed using human mitochondrial DNA restriction site data. \\

A restriction haploptype is the combination of presence (1) 
or absence (0) of restriction sites with length S. \\

It can be present as a vector (p is either 1 or 0):

\[
\begin{bmatrix}
    p_1 \\
    p_2 \\
    p_3 \\
    \vdots \\
    p_S 
\end{bmatrix}
\]

A linear model is set up as:

\begin{equation*}
    \boldsymbol{p_{jig}} = \boldsymbol{p} + \boldsymbol{a_g} + \boldsymbol{b_{ig}} + \boldsymbol{c_{jig}}
\end{equation*}

where j represents the haplotype, i represents the population, g represents the group. \\

Example:
There are two populations with two individuals in each population. \\

A system of equations is written as:

\begin{align*}
        \boldsymbol{p_{11}} &= \boldsymbol{p} + \boldsymbol{b_{1}} + \boldsymbol{\epsilon} \\
        \boldsymbol{p_{12}} &= \boldsymbol{p} + \boldsymbol{b_{1}} + \boldsymbol{\epsilon} \\
        \boldsymbol{p_{21}} &= \boldsymbol{p} + \boldsymbol{b_{2}} + \boldsymbol{\epsilon} \\
        \boldsymbol{p_{22}} &= \boldsymbol{p} + \boldsymbol{b_{2}} + \boldsymbol{\epsilon}
\end{align*}

Written as matrix form (each value is a column vector of length S):

\[
\underbrace{
\begin{bmatrix}
    \boldsymbol{p_{11}} \\
    \boldsymbol{p_{12}} \\
    \boldsymbol{p_{21}} \\
	\boldsymbol{p_{22}}
\end{bmatrix}
}_\text{observed values of each individual}
= 
\underbrace{
\begin{bmatrix}
    \boldsymbol{1} & \boldsymbol{1} & \boldsymbol{0} \\
    \boldsymbol{1} & \boldsymbol{1} & \boldsymbol{0} \\
    \boldsymbol{1} & \boldsymbol{0} & \boldsymbol{1} \\
    \boldsymbol{1} & \boldsymbol{0} & \boldsymbol{1} \\
\end{bmatrix} 
}_\text{a matrix with linear dependent columns}
\begin{bmatrix}
    \boldsymbol{p} \\
    \boldsymbol{b_1} \\
    \boldsymbol{b_2}
\end{bmatrix}
+ 
\begin{bmatrix}
    \boldsymbol{\epsilon_1} \\
    \boldsymbol{\epsilon_2} \\
    \boldsymbol{\epsilon_3} \\
    \boldsymbol{\epsilon_4}
\end{bmatrix}
\]

The matrix with 1 and 0 is used to form the X matrix in general linear model by removing redundant columns. \\

Two examples of X matrix: \\

Remove the third column, X matrix is
\[
\boldsymbol{X}
=
\begin{bmatrix}
	\boldsymbol{1} & \boldsymbol{0} \\
	\boldsymbol{1} & \boldsymbol{0} \\
   \boldsymbol{1} & \boldsymbol{1} \\
	\underbrace{\boldsymbol{1}}_{\text{intercept}} & \boldsymbol{1} \\
\end{bmatrix}
\]
The column with all 1 is called the intercept. \\

Remove the first column, X matrix is
\[
\boldsymbol{X} =
\begin{bmatrix}
	\boldsymbol{1} & \boldsymbol{0} \\
	\boldsymbol{1} & \boldsymbol{0} \\
	\boldsymbol{0} & \boldsymbol{1} \\
	\boldsymbol{0} & \boldsymbol{1} \\
\end{bmatrix}
\]

The two X matrices are row equivalent. \\ 

New X matrices can also be obtained using elementary row operations. \\

For example, 

\begin{equation*} \begin{split}
\begin{bmatrix}
	1 & 0 \\
	1 & 0 \\
	0 & 1 \\
	0 & 1 
\end{bmatrix}
\xrightarrow{\text{$R1 \rightarrow R1-R3$}}
\begin{bmatrix}
	1 & -1 \\
	1 & 0 \\
	0 & 1 \\
	0 & 1 
\end{bmatrix}
\xrightarrow{\text{$R2 \rightarrow R2-R3$}}
\begin{bmatrix}
	1 & -1 \\
	1 & -1 \\
	0 & 1 \\
	0 & 1 
\end{bmatrix}
\xrightarrow[\text{$R4 \rightarrow R4 \times 2$}]{\text{$R3 \rightarrow R3 \times 2$}}
\begin{bmatrix}
	1 & -1 \\
	1 & -1 \\
	0 & 2 \\
	0 & 2
\end{bmatrix}
\xrightarrow[\text{$R4 \rightarrow R4+R2$}]{\text{$R3 \rightarrow R3+R1$}}
\begin{bmatrix}
	1 & -1 \\
	1 & -1 \\
	1 & 1 \\
	1 & 1
\end{bmatrix}
\end{split} \end{equation*}

The row equivalent X matrix is $\begin{bmatrix}
	1 & -1 \\
	1 & -1 \\
	1 & 1 \\
	1 & 1
\end{bmatrix}$. \\

Different X matrices give different equations: 

\[
\underbrace{
\begin{bmatrix}
	\boldsymbol{1} & \boldsymbol{0} \\
	\boldsymbol{1} & \boldsymbol{0} \\
   \boldsymbol{1} & \boldsymbol{1} \\
	\boldsymbol{1} & \boldsymbol{1} \\
\end{bmatrix}
}_\text{$\boldsymbol{X}$}
\underbrace{
\begin{bmatrix}
	\boldsymbol{p_1} \\
 	\boldsymbol{p_2} - \boldsymbol{p_1} 
\end{bmatrix}
}_\text{$\boldsymbol{\beta}$}
= 
\underbrace{
\begin{bmatrix}
	\boldsymbol{p_1} \\
	\boldsymbol{p_1} \\
	\boldsymbol{p_2} \\
	\boldsymbol{p_2}
\end{bmatrix}
}_\text{predicted values}
\]

\[
\begin{bmatrix}
	\boldsymbol{1} & \boldsymbol{0} \\
	\boldsymbol{1} & \boldsymbol{0} \\
	\boldsymbol{0} & \boldsymbol{1} \\
	\boldsymbol{0} & \boldsymbol{1} \\
\end{bmatrix}
\begin{bmatrix}
	\boldsymbol{p_1} \\
	\boldsymbol{p_2} 
\end{bmatrix}
= 
\begin{bmatrix}
	\boldsymbol{p_1} \\
	\boldsymbol{p_1} \\
	\boldsymbol{p_2} \\
	\boldsymbol{p_2}
\end{bmatrix}
\]

$\boldsymbol{p_1}$ and $\boldsymbol{p_2}$ are means of populations 1 and 2. \\

Note that they have different X matrices and $\boldsymbol{\beta}$ vector. But they give the same predicted values.\\


Choosing the columns of X matrix is called \textbf{contrasts}.
\end{document}
