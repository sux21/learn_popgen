\documentclass[12pt]{article}
\title{General Linear Model}
\author{Xingyuan Su}
\date{\today}

\usepackage{amsmath}

\setlength{\parindent}{0pt} % don't indent the line
\begin{document}

\maketitle

\section{Expression}

General linear model is expressed as:

\begin{equation*}
    \boldsymbol{Y} = \boldsymbol{X}\boldsymbol{\beta} + \boldsymbol{\epsilon}
\end{equation*}

The matrix form of the equation:

\[    
\begin{bmatrix} 
    Y_1 \\ 
    Y_2 \\ 
    Y_3 \\ 
    \vdots \\
    Y_n
\end{bmatrix}_{n \times 1}
= 
\begin{bmatrix}
    x_{11} & x_{12} & x_{13} & \cdots & x_{1p} \\
    x_{21} & x_{22} & x_{23} & \cdots & x_{2p} \\
    x_{31} & x_{32} & x_{33} & \cdots & x_{3p} \\
    \vdots & \vdots & \vdots & \ddots & \vdots \\
    x_{n1} & x_{n2} & x_{n3} & \cdots & x_{np}
\end{bmatrix}_{n \times p}
\begin{bmatrix}
    \beta_1 \\
    \beta_2 \\
    \beta_3 \\
    \vdots \\
    \beta_p 
\end{bmatrix}_{p \times 1}
+ 
\begin{bmatrix}
    \epsilon_1 \\
    \epsilon_2 \\
    \epsilon_3 \\
    \vdots \\
    \epsilon_n
\end{bmatrix}_{n \times 1}
\]

The goal of general linear model is to find $\boldsymbol{\beta}$ that minimizes 
the distance between $\boldsymbol{Y}$ and $\boldsymbol{X\beta}$. 

\section{Least Squares Estimate of $\boldsymbol{\beta}$}
\begin{equation*}
	\boldsymbol{\beta} = (\boldsymbol{X^T}\boldsymbol{X})^{-1}\boldsymbol{X}\boldsymbol{Y}
\end{equation*}

\section{Examples}
Consider a system of equations:

\begin{align*}
        p_{11} &= p + b_{1} + \epsilon \\
        p_{12} &= p + b_{1} + \epsilon \\
        p_{21} &= p + b_{2} + \epsilon \\
        p_{22} &= p + b_{2} + \epsilon
\end{align*}

Express it in matrix form:

\[
\underbrace{
\begin{bmatrix}
    p_{11} \\
    p_{12} \\
    p_{21} \\
    p_{22}
\end{bmatrix}
}_\text{observed values}
= 
\underbrace{
\begin{bmatrix}
    1 & 1 & 0 \\
    1 & 1 & 0 \\
    1 & 0 & 1 \\
    1 & 0 & 1 \\
\end{bmatrix} 
}_\text{a matrix with linear dependent columns}
\begin{bmatrix}
    p \\
    b_1 \\
    b_2
\end{bmatrix}
+ 
\begin{bmatrix}
    \epsilon_1 \\
    \epsilon_2 \\
    \epsilon_3 \\
    \epsilon_4
\end{bmatrix}
\]

The matrix with 1 and 0 is used to form the X matrix in general linear model by removing redundant columns. \\

For example, remove the third column, X matrix is
\[
\boldsymbol{X}
=
\begin{bmatrix}
	1 & 0 \\
	1 & 0 \\
   1 & 1 \\
	\underbrace{1}_{\text{intercept}} & 1 \\
\end{bmatrix}
\]
The column with all 1 is called the intercept. \\

Or, remove the first column, X matrix is
\[
\boldsymbol{X} =
\begin{bmatrix}
	1 & 0 \\
	1 & 0 \\
	0 & 1 \\
	0 & 1 \\
\end{bmatrix}
\]

The two X matrices are row equivalent. \\ 

New X matrices can also be obtained using elementary row operations. \\

For example, 

\begin{equation*}
\begin{bmatrix}
	1 & 0 \\
	1 & 0 \\
	0 & 1 \\
	0 & 1 
\end{bmatrix}
\xrightarrow{\text{$R1 \rightarrow R1-R3$}}
\begin{bmatrix}
	1 & -1 \\
	1 & 0 \\
	0 & 1 \\
	0 & 1 
\end{bmatrix}
\xrightarrow{\text{$R2 \rightarrow R2-R3$}}
\begin{bmatrix}
	1 & -1 \\
	1 & -1 \\
	0 & 1 \\
	0 & 1 
\end{bmatrix}
\xrightarrow[\text{$R4 \rightarrow R4 \times 2$}]{\text{$R3 \rightarrow R3 \times 2$}}
\begin{bmatrix}
	1 & -1 \\
	1 & -1 \\
	0 & 2 \\
	0 & 2
\end{bmatrix}
\end{equation*}

\begin{equation*}
\xrightarrow[\text{$R4 \rightarrow R4+R2$}]{\text{$R3 \rightarrow R3+R1$}}
\begin{bmatrix}
	1 & -1 \\
	1 & -1 \\
	1 & 1 \\
	1 & 1
\end{bmatrix}
\end{equation*}

The new X matrix is $\begin{bmatrix}
	1 & -1 \\
	1 & -1 \\
	1 & 1 \\
	1 & 1
\end{bmatrix}$. \\

Different X matrices give different equations: 

\[
\underbrace{
\begin{bmatrix}
	1 & 0 \\
	1 & 0 \\
   1 & 1 \\
   1 & 1 \\
\end{bmatrix}
}_\text{$\boldsymbol{X}$}
\underbrace{
\begin{bmatrix}
	p_1 \\
 	p_2 - p_1
\end{bmatrix}
}_\text{$\boldsymbol{\beta}$}
= 
\underbrace{
\begin{bmatrix}
	p_1 \\
	p_1 \\
	p_2 \\
   p_2
\end{bmatrix}
}_\text{predicted values}
\]

\[
\underbrace{
\begin{bmatrix}
	1 & 0 \\
	1 & 0 \\
	0 & 1 \\
   0 & 1 \\
\end{bmatrix}
}_\text{$\boldsymbol{X}$}
\underbrace{
\begin{bmatrix}
	p_1 \\
	p_2
\end{bmatrix}
}_\text{$\boldsymbol{\beta}$}
= 
\underbrace{
\begin{bmatrix}
	p_1 \\
	p_1 \\
	p_2 \\
	p_2
\end{bmatrix}
}_\text{predicted values}
\]

Note that different X matrices have different $\boldsymbol{\beta}$ vector. But they have the same predicted values.\\


Choosing an X matrix is called \textbf{contrast}.
\end{document}
